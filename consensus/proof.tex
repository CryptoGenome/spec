\section{Proof of Tendermint consensus algorithm}
\label{sec:proof}

\begin{lemma}
	\label{lemma:majority-intersection}
	For all $f\geq 0$, any two sets of processes with voting power at least equal to $2f+1$ have
	at least one correct process in common.
\end{lemma}

\begin{proof}
	As the total voting power is equal to $n=3f+1$, we have $2(2f+1) = n+f+1$.
	This means that the intersection of two sets with the voting power equal to $2f+1$ contains at least $f+1$ voting power in common, \ie, at least one correct process (as the total voting power of faulty processes is $f$). The result follows directly from this.
\end{proof}

\begin{lemma}
	\label{lemma:locked-decision_value-prevote-v}
	If $f+1$ correct processes locks value $v$ in round $r_0$ ($lockedValue = v$ and $lockedRound = r_0$), then in all rounds $r > r_0$, they send $\Prevote$ for $id(v)$ or $\nil$.
\end{lemma}

\begin{proof}
We prove the result by induction on $r$.

\emph{Base step $r = r_0 + 1:$} Lets denote with $C$ the set of correct processes with voting power equal to $f+1$. 
By the rules at line~\ref{line:tab:recvProposal} and line~\ref{line:tab:acceptProposal}, the processes from the set $C$ can't accept $\Proposal$ for any value different from $v$ in round $r$, and therefore can't send a $\li{\Prevote,height_p, r,id(v')}$ message,
if $v' \neq v$. Therefore, the Lemma holds for the base step.

\emph{Induction step from $r_1$ to $r_1+1$:} We assume that no process from the set $C$ has sent $\Prevote$ for values different than $id(v)$ or $\nil$ until round $r_1 + 1$. We now prove that 
the Lemma also holds for round $r_1 + 1$. As processes from the set $C$ send $\Prevote$ for $id(v)$ or $\nil$ in rounds $r_0 \le r \le r_1$, by Lemma~\ref{lemma:majority-intersection}
there is no value $v' \neq v$ for which it is possible to receive $2f+1$ $\Prevote$ messages in those rounds (i). Therefore, we have for all processes from the set $C$, $lockedValue = v$ and $lockedRound \ge r_0$.   
Lets assume by a contradiction that a process $q$ from the set $C$ sends $\Prevote$ in round $r_1 + 1$ for value $id(v')$, where $v' \neq v$. This is possible only by line~\ref{line:tab:prevote-higher-proposal}. 
Note that this implies that $q$ received $2f+1$ $\li{\Prevote,h_q, r,id(v')}$ messages, where $r > r_0$ and $r < r_1 +1$ (see line~\ref{line:tab:cond-prevote-higher-proposal}). A contradiction with (i) and Lemma~\ref{lemma:majority-intersection}.  	
\end{proof}	

\begin{lemma}
	\label{lemma:agreement}
	Algorithm~\ref{alg:tendermint} satisfies Agreement. 
\end{lemma}

\begin{proof}
Let round $r_0$ be the first round of height $h$ such that some correct process $p$ decides $v$. We now prove that if some correct process $q$ decides $v'$ in some round $r \ge r_0$, then $v = v'$.

In case $r = r_0$, $q$ has received at least $2f+1$ $\li{\Precommit,h_p,r_0,id(v')}$  messages at line~\ref{line:tab:onDecideRule},
while $p$ has received at least $2f+1$ $\li{\Precommit,h_p,r_0,id(v)}$ messages. 
By Lemma~\ref{lemma:majority-intersection} two sets of messages of voting power $2f+1$ intersect in at least one correct process.
As a correct process sends a single $\Precommit$ message in a round, then $v=v'$.

We prove the case $r > r_0$ by contradiction. By the rule~\ref{line:tab:onDecideRule}, $p$ has received at least $2f+1$ voting-power equivalent of $\li{\Precommit,h_p,r_0,id(v)}$ messages, i.e., at least $f+1$ voting-power equivalent processes 
have locked value $v$ in round $r_0$ and have sent those messages (i). Let denote this set of messages with $C$.
On the other side, $q$ has received at least $2f+1$ voting power equivalent of $\li{\Precommit,h_q, r,id(v')}$ messages. As the voting power of all faulty processes is at most $f$, some correct process $c$ has sent one of those messages. By the rule at line~\ref{line:tab:recvPrevote}, $c$ has locked value $v'$ in round $r$ before sending $\li{\Precommit,h_q, r,id(v')}$. Therefore $c$ has received $2f+1$ $\Prevote$ messages for $id(v')$ in round $r > r_0$ (see line~\ref{line:tab:recvPrevote}). By Lemma~\ref{lemma:majority-intersection}, a process from the set $C$ has sent $\Prevote$ message for $id(v')$ in round $r$. 
A contradiction with (i) and Lemma~\ref{lemma:locked-decision_value-prevote-v}.  
\end{proof}	

\begin{lemma}
	\label{lemma:agreement}
	Algorithm~\ref{alg:tendermint} satisfies Validity. 
\end{lemma}

\begin{proof}
Trivially follows from the rule at line \ref{line:tab:validDecisionValue} which ensures that only valid values can be decided. 
\end{proof}	

\begin{lemma}
	\label{lemma:round-synchronisation}
If we assume that:
\begin{enumerate}
	\item a correct process $p$ is the first correct process to enter a round $r>0$ at time $t > GST$ (for every correct process $c$, $round_c \le r$ at time $t$)
	\item a proposer of round $r$ is a correct process $q$ 
	\item for every correct process $c$, $lockedRound_c \le validRound_q$ at time $t$
	\item $\timeoutPropose > 2\Delta + \timeoutPrecommit$, $\timeoutPrevote > 2\Delta$ and $\timeoutPrecommit > 2\Delta$,
\end{enumerate}
then all correct processes decide in round $r$ before $t + 4\Delta + \timeoutPrecommit$.  
\end{lemma}	

\begin{proof}
As $p$ is the first correct process to have entered round $r$, it executed the line~\ref{line:tab:nextRound} after $\timeoutPrecommit$ expired. Therefore, $p$ received $2f+1$ $\Precommit$ messages in round $r-1$ before time $t$. By the \emph{Gossip communication} property, all correct processes will receive those messages by time $t + \Delta$ in the latest. Correct processes that
are in rounds $< r-1$ at time $t$ will enter round $r-1$ (see the rule at line~\ref{line:tab:nextRound2}) and trigger $\timeoutPrecommit$ (see rule~\ref{line:tab:startTimeoutPrecommit})
by time $t+\Delta$. Therefore, all correct processes will start round $r$ by time $t+\Delta+\timeoutPrecommit$ (i).
 
In the worst case, the process $q$ is the last correct process to enter round $r$, so $q$ starts round $r$ and sends $\Proposal$ message for some value $v$ at time $t + \Delta + \timeoutPrecommit$. Therefore, all correct processes receive the $\Proposal$ message from $q$ the latest by time $t + 2\Delta + \timeoutPrecommit$. Therefore, if $\timeoutPropose > 2\Delta + \timeoutPrecommit$, 
all correct processes will receive $\Proposal$ message before $\timeoutPropose$ expires. 

By (3) and the rules at line~\ref{line:tab:recvProposal} and \ref{line:tab:acceptProposal}, all correct processes will accept the $\Proposal$ message for value $v$ and will send a $\Prevote$ message for $id(v)$ by time $t + 2\Delta + \timeoutPrecommit$. 
Note that by the \emph{Gossip communication} property, the $\Prevote$ messages needed to trigger the rule at line~\ref{line:tab:acceptProposal} are received before time $t + \Delta$.  

By time $t + 3\Delta + \timeoutPrecommit$, all correct processes will receive $\Proposal$ for $v$ and $2f+1$ corresponding $\Prevote$ messages for $id(v)$. By the rule at line~\ref{line:tab:recvPrevote}, all correct processes will send a $\Precommit$ message (see line~\ref{line:tab:precommit-v}) for $id(v)$ by time $t + 3\Delta + \timeoutPrecommit$. Therefore, by time $t + 4\Delta + \timeoutPrecommit$, all correct processes will have received the $\Proposal$ for $v$ and $2f+1$ $\Precommit$ messages for $id(v)$, so they decide at line~\ref{line:tab:decide} on $v$. 

This scenario holds if every correct process $q$ sends a $\Precommit$ message before $\timeoutPrevote$ expires, and if $\timeoutPrecommit$ does not expire before $t + 4\Delta + \timeoutPrecommit$. 
Lets assume that a correct process $c_1$ is the first correct process to trigger $\timeoutPrevote$ (see the rule at line~\ref{line:tab:recvAny2/3Prevote}) at time $t_1 > t$. This implies that before time $t_1$, $c_1$ received a $\Proposal$ ($step_{c_1}$ must be $\prevote$ by the rule at line~\ref{line:tab:recvAny2/3Prevote}) and a set of $2f+1$ $\Prevote$ messages.
By time $t_1 + \Delta$, all correct processes will receive those messages. Note that even if some correct process was in the smaller round before time $t_1$, at time $t_1 + \Delta$ it will start round $r$ after receiving those messages (see the rule at line~\ref{line:tab:skipRounds}). 
Therefore, all correct processes will send their $\Prevote$ message for $id(v)$ by time $t_1 + \Delta$, and all correct processes will receive those messages the by time $t_1 + 2\Delta$. 
Therefore, as $\timeoutPrevote > 2\Delta$, this ensures that all correct processes receive $\Prevote$ messages from all correct processes before their respective local $\timeoutPrevote$ expire.   

On the other hand, $\timeoutPrecommit$ is triggered in a correct process $c_2$ after it receives any set of $2f+1$ $\Precommit$ messages for the first time. Lets denote with $t_2 > t$ the earliest point in time $\timeoutPrecommit$ is triggered in some correct process $c_2$. This implies that $c_2$ has received at least $f+1$ $\Precommit$ messages for $id(v)$ from correct processes, i.e., those processes have received $\Proposal$ for $v$ and $2f+1$ $\Prevote$ messages for $id(v)$ before time $t_2$. By the \emph{Gossip communication} property, all correct processes will receive those messages by time $t_2 + \Delta$, and will send $\Precommit$ messages for $id(v)$. Note that even if some correct processes were at time $t_2$ in a round smaller than $r$,
by the rule at line~\ref{line:tab:skipRounds} they will enter round $r$ by time $t_2 + \Delta$.
Therefore, by time $t_2 + 2\Delta$, all correct processes will receive $\Proposal$ for $v$ and $2f+1$ $\Precommit$ messages for $id(v)$. So if $\timeoutPrecommit > 2\Delta$, all correct processes will decide before the timeout expires.         
\end{proof}	


\begin{lemma}
	\label{lemma:validValue}
If a correct process $p$ locks a value $v$ at time $t_0 > GST$ in some round $r$ ($lockedValue = v$ and $lockedRound = r$) and $\timeoutPrecommit> 2\Delta$, then all correct processes set $validValue$ to $v$ and $validRound$ to $r$ before starting round $r+1$. 
\end{lemma}
 
\begin{proof}
In order to prove this Lemma, we need to prove that if the process $p$ locks a value $v$ at time $t_0$, then no correct process will leave round $r$ before time $t_0 + \Delta$ (unless it has already set $validValue$ to $v$ and $validRound$ to $r$). It is sufficient to prove this, since by the \emph{Gossip communication} property the messages that $p$ received at time $t$ and that triggered rule at line~\ref{line:tab:recvPrevote} will be received by time $t_0 + \Delta$ by all correct processes, so all correct processes that are still in round $r$ will set $validValue$ to $v$ and $validRound$ to $r$ (by the rule at line~\ref{line:tab:recvPrevote}). To prove this, we need to compute the earliest point in time a correct process could leave round $r$ without updating $validValue$ to $v$ and $validRound$ to $r$ (we denote this time with $t_1$). The Lemma is correct if $t_0 + \Delta < t_1$. 

If the process $p$ locks a value $v$ at time $t$, this implies that $p$ received the valid $\Proposal$ message for $v$ and $2f+1$ $\li{\Prevote,h,r,id(v)}$ at time $t$. At least $f+1$ of those messages are sent by correct processes. Lets denote this set of correct processes as $C$. By Lemma~\ref{lemma:majority-intersection} any set of $2f+1$ $\Prevote$ messages in round $r$ contains at least a single message from the set $C$. 

Lets denote as time $t$ the earliest point in time a correct process triggered $\timeoutPrevote$. This implies that $c$ received $2f+1$ $\Prevote$ messages
(see the rule at line \ref{line:tab:recvAny2/3Prevote}), where at least one of those messages was sent by a process $c$ from the set $C$.  Therefore, process $c$ had received $\Proposal$ message before time $t$. By the \emph{Gossip communication} property, all correct processes will receive $\Proposal$ and $2f+1$ $\Prevote$ messages for round $r$ by time $t+\Delta$. The latest point in time $p$ will trigger $\timeoutPrevote$ is $t+\Delta$\footnote{Note that even if $p$ was in smaller round at time $t$ it will start round $r$ by time $t+\Delta$.}.  So the latest point in time $p$ can lock the value value $v$ in round $r$ is $t_0 = t+\Delta+\timeoutPrevote$ (as at this point $\timeoutPrevote$ expires, so a process sends $\Precommit$ $\nil$ and updates $step$ to $\precommit$, see line \ref{line:tab:onTimeoutPrevote}).  

Note that according to the algorithm \ref{alg:tendermint}, a correct process can not send a $\Precommit$ message before receiving $2f+1$ $\Prevote$ messages.  Therefore, no correct process can send a $\Precommit$ message in round $r$ before time $t$. If a correct process sends a $\Precommit$ message for $\nil$, it implies that it has waited for the full duration of $\timeoutPrevote$ (see line \ref{line:tab:precommit-nil-onTimeout})\footnote{The other case in which a correct process $\Precommit$ for $\nil$ is after receiving $2f+1$ $Prevote$ for $\nil$ messages, see the line \ref{line:tab:precommit-v-1}. By Lemma~\ref{lemma:majority-intersection}, this is not possible in round $r$.}. Therefore, no correct process can send $\Precommit$ for $\nil$ before time $t + \timeoutPrevote$ (*).

A correct process $q$ that enters round $r+1$ must have waited (i) $\timeoutPrecommit$ (see line \ref{line:tab:nextRound}) or (ii) received $f+1$ messages from round $r+1$ (see line \ref{line:tab:skipRounds}). 
In the former case, $q$ received $2f+1$ $\Precommit$ messages before starting $\timeoutPrecommit$. If at least a single $\Precommit$ message from a correct process (at least $f+1$ voting power equivalent of those messages was sent by correct processes) is for $\nil$, then $q$ could not have started round $r+1$ before time $t_1 = t + \timeoutPrevote + \timeoutPrecommit$ (see (*)). Therefore in this case we have: $t_0 + \Delta < t_1$, i.e., 
$t+2\Delta+\timeoutPrevote <  t + \timeoutPrevote + \timeoutPrecommit$, and this is true whenever $\timeoutPrecommit > 2\Delta$, so Lemma holds.

If in the set of $2f+1$ $\Precommit$ messages $q$ received, there was at least a single $\Precommit$ message for $id(v)$ from a correct process $c$, 
then $q$ could have started round $r+1$ the earliest by time $t_1 = t+\timeoutPrecommit$. In this case,
by the \emph{Gossip communication} property, all correct processes will receive the $\Proposal$ and $2f+1$ $\Prevote$ messages (that $c$ received before time $t$) by time $t+\Delta$. Therefore, $q$ will set $validValue$ to $v$ and $validRound$ to $r$ by time $t+\Delta$. As $t+\Delta < t+\timeoutPrecommit$, whenever $\timeoutPrecommit > \Delta$, the Lemma holds also in this case.    

In case (ii), $q$ received at least a single message from a correct process $c$ from the round
$r+1$. The earliest point in time $c$ could have started round $r+1$ is $t+\timeoutPrecommit$ in case it received a $\Precommit$ message for $v$ from some correct process in the set of $2f+1$ $\Precommit$ messages it received. The same reasoning as above holds also in this case, so $q$ set $validValue$ to $v$ and $validRound$ to $r$ the latest by time $t+\Delta$. As $t+\Delta < t+\timeoutPrecommit$, whenever $\timeoutPrecommit > \Delta$, the Lemma holds also in this case.    
\end{proof}

\begin{lemma}
	\label{lemma:agreement}
Algorithm~\ref{alg:tendermint} satisfies Termination. 
\end{lemma}

\begin{proof}
Lemma~\ref{lemma:round-synchronisation} defines a scenario in which all correct processes decide. We now prove that within a bounded duration after GST such a scenario will unfold. Lets
assume that at time $GST$ the highest round started by a correct process is $r_0$, and that there exists a correct process $p$ such that the following holds: for every correct process $c$, $lockedRound_c \le validRound_p$. Furthermore, we assume that $p$ will be the proposer in some round $r_1 > r$ (this is ensured by the $\coord$ function). 

We have two cases to consider. In the first case, for all rounds $r \ge r_0$ and $r < r_1$, no correct process locks a value (set $lockedRound$ to $r$). So in round $r_1$ we have the scenario from the Lemma~\ref{lemma:round-synchronisation}, so all correct processes decides in round $r_1$.  

In the second case, a correct process locks a value $v$ in round $r_2$, where $r_2 \ge r_0$ and $r_2 < r_1$.  Lets assume that $r_2$ is the highest round before $r_1$ in which some correct process $q$ locks a value. By Lemma \ref{lemma:validValue} at the end of round $r_2$ the following holds for all correct processes $c$: $validValue_c = lockedValue_q$ and $validRound_c = r_2$. Then in round $r_1$, the conditions for the Lemma~\ref{lemma:round-synchronisation} holds, so all correct processes decide.
\end{proof}	

