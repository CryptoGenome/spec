\section{Definitions}
\label{sec:definitions}

\subsection{Model}

We consider a system of processes that communicate by exchanging messages.
Processes can be correct or faulty, where a faulty process can behave in an arbitrary way,
 i.e., we consider Byzantine faults. We assume that each process has some amount of voting power (voting power of a process can be $0$).
Processes in our model are not part of the single administration domain; therefore we cannot enforce a direct network connectivity between all processes. Instead, we assume that each process is connected to a subset of processes called peers, 
such that there is an indirect communication channel between all correct processes. Communication between processes is established using a gossip protocol \cite{Dem1987:gossip}.

Formally, we model the network communication using the \emph{partially synchronous system model}~\cite{DLS88:jacm}: in all executions
of the system there is a bound $\Delta$ and an instant GST (Global Stabilization Time) such that all communication among correct processes 
after GST is reliable and $\Delta$-timely, i.e., if a correct process $p$ sends message $m$ at time $t \ge GST$ to correct process $q$, then $q$ will receive $m$ before $t + \Delta$\footnote{Note that as we do not assume direct communication channels among all correct processes, this implies that before the message $m$ reaches $q$, it might pass through a number of 
	correct processes that will forward the message $m$ using gossip protocol towards $q$.}. Messages among correct processes can be delayed, dropped or duplicated before GST. 
The bound $\Delta$ and GST are system parameters whose values are not required to be known for the safety of our algorithm. Termination of the algorithm is guaranteed within a bounded duration after GST.
In practice, the algorithm will work correctly in the slightly weaker variant of the model where the system alternates between (long enough) good periods (corresponds to the \emph{after} GST period where system is reliable and $\Delta$-timely) and bad periods (corresponds to the period \emph{before} GST during which the system is asynchronous and messages can be lost), but considering the GST model simplifies the discussion.  

We assume that process steps (which might include sending and receiving messages) take zero time.
Processes are equipped with clocks so they can measure local timeouts.
All protocol messages are signed, i.e., when a correct process $q$ receives a signed message $m$ from its peer, the process $q$ can verify who was the original sender of the message $m$.

The details of the Tendermint gossip protocol will be discussed in a separate technical report. For the sake of this report it is sufficient to assume that messages are being gossiped between processes and the following property holds (in addition to the partial synchrony network assumptions):

\begin{itemize}
	\item \emph{Gossip communication:} If a correct process $p$ receives some message $m$ at time $t$, all correct processes will receive $m$ before $max\{t, GST\} + \Delta$.    
\end{itemize}


%Messages that are being gossiped are created by the consensus layer. We can think about consensus protocol as a content creator, which %defines what messages should be disseminated using the gossip protocol. A correct process creates the message for dissemination either i) %explicitly, by invoking \emph{send} function as part of the consensus protocol or ii) implicitly, by receiving a message from some other %process. Note that in the case ii) gossiping of messages is implicit, i.e., it happens without explicit send clause in the consensus algorithm %whenever a correct process receives some messages in the consensus algorithm\footnote{If a message is received by a correct process at %the consensus level then it is considered valid from the protocol point of view, i.e., it has a correct signature, a proper message structure %and a valid height and round number.}. 

%\item Processes keep resending messages (in case of failures or message loss) until all its peers get them. This ensures that every message %sent or received by a correct process is eventually received by all correct processes. 

\subsection{State Machine Replication}

State machine replication (SMR) is a general approach for replicating
services modeled as a deterministic state
machine~\cite{Lam78:cacm,Sch90:survey}.
The key idea of this approach is to guarantee that all replicas start
in the same state and then apply requests from clients in the
same order, thereby guaranteeing that the replicas' states will
not diverge.
Following Schneider~\cite{Sch90:survey}, we note that the following is key for
implementing a replicated state machine tolerant to (Byzantine) faults:

\begin{itemize}
	\item \emph{Replica Coordination.} All [non-faulty] replicas receive
	and process the same sequence of requests.
\end{itemize}

Moreover, as Schneider also notes this property can be decomposed into
two parts, \emph{Agreement} and \emph{Order}: Agreement
requires all (non-faulty) replicas to receive all
requests, and Order requires that the order of received requests
is the same at all replicas.

There is an additional requirement that needs to be ensured by Byzantine tolerant state machine replication:
only requests (called transactions in the Tendermint terminology) proposed by clients are executed. In Tendermint, transaction verification is responsibility of the service that is 
being replicated; upon receiving a transaction from the client, the Tendermint process will ask the service if the request is valid, and only valid 
requests will be processed. 

 \subsection{Consensus}
 \label{sec:consensus}

Tendermint solves state machine replication by sequentially executing consensus instances to agree on the content of the next block of transactions that are then executed by the service being replicated. We consider a variant of the Byzantine consensus problem called Validity Predicate-based Byzantine consensus that is motivated by blockchain
systems~\cite{GLR17:red-belly-bc}. The problem is defined by an agreement, a termination, and a validity
property.

 \begin{itemize}
 \item \emph{Agreement:} No two correct processes decide on different values.
 \item \emph{Termination:} All correct processes eventually decide on a value.
 \item \emph{Validity:} A decided value is valid, i.e., it satisfies the predefined predicate denoted \emph{valid()}.
 \end{itemize}

This variant of the Byzantine consensus problem has an application-specific valid() predicate to indicate whether a value is valid. In the context of blockchain systems, for example, a value is not valid if it does not contain an appropriate hash of the last value (block) added to the blockchain.
