%\documentclass[conference]{IEEEtran}
\documentclass[conference,onecolumn,draft,a4paper]{IEEEtran}
% Add the compsoc option for Computer Society conferences.
%
% If IEEEtran.cls has not been installed into the LaTeX system files,
% manually specify the path to it like:
% \documentclass[conference]{../sty/IEEEtran}



% *** GRAPHICS RELATED PACKAGES ***
%
\ifCLASSINFOpdf
\else
\fi

% correct bad hyphenation here
\hyphenation{op-tical net-works semi-conduc-tor}

%\usepackage[caption=false,font=footnotesize]{subfig}
\usepackage{tikz}
\usetikzlibrary{decorations,shapes,backgrounds,calc}
\tikzstyle{msg}=[->,black,>=latex]
\tikzstyle{rubber}=[|<->|]
\tikzstyle{announce}=[draw=blue,fill=blue,shape=diamond,right,minimum
  height=2mm,minimum width=1.6667mm,inner sep=0pt]
\tikzstyle{decide}=[draw=red,fill=red,shape=isosceles triangle,right,minimum
  height=2mm,minimum width=1.6667mm,inner sep=0pt,shape border rotate=90]
\tikzstyle{cast}=[draw=green!50!black,fill=green!50!black,shape=circle,left,minimum
  height=2mm,minimum width=1.6667mm,inner sep=0pt]


\usepackage{multirow}
\usepackage{graphicx}
\usepackage{epstopdf}
\usepackage{amssymb}
\usepackage{rounddiag}
\graphicspath{{../}}

\usepackage{technote}
\usepackage{homodel}
\usepackage{enumerate}
%%\usepackage{ulem}\normalem

% to center caption
\usepackage{caption}

\newcommand{\textstretch}{1.4}
\newcommand{\algostretch}{1}
\newcommand{\eqnstretch}{0.5}

\newconstruct{\FOREACH}{\textbf{for each}}{\textbf{do}}{\ENDFOREACH}{}

%\newconstruct{\ON}{\textbf{on}}{\textbf{do}}{\ENDON}{\textbf{end on}}
\newcommand\With{\textbf{while}}
\newcommand\From{\textbf{from}}
\newcommand\Broadcast{\textbf{broadcast}}
\newcommand\PBroadcast{send}
\newcommand\UpCall{\textbf{UpCall}}
\newcommand\DownCall{\textbf{DownCall}}
\newcommand \Call{\textbf{Call}}
\newident{noop}
\newconstruct{\UPON}{\textbf{upon}}{\textbf{do}}{\ENDUPON}{}



\newcommand{\abcast}{\mathsf{to\mbox{\sf-}broadcast}}
\newcommand{\adeliver}{\mathsf{to\mbox{\sf-}deliver}}

\newcommand{\ABCAgreement}{\emph{TO-Agreement}}
\newcommand{\ABCIntegrity}{\emph{TO-Integrity}}
\newcommand{\ABCValidity}{\emph{TO-Validity}}
\newcommand{\ABCTotalOrder}{\emph{TO-Order}}
\newcommand{\ABCBoundedDelivery}{\emph{TO-Bounded Delivery}}


\newcommand{\tabc}{\mathit{atab\mbox{\sf-}cast}}
\newcommand{\anno}{\mathit{atab\mbox{\sf-}announce}}
\newcommand{\abort}{\mathit{atab\mbox{\sf-}abort}}
\newcommand{\tadel}{\mathit{atab\mbox{\sf-}deliver}}

\newcommand{\ATABAgreement}{\emph{ATAB-Agreement}}
\newcommand{\ATABAbort}{\emph{ATAB-Abort}}
\newcommand{\ATABIntegrity}{\emph{ATAB-Integrity}}
\newcommand{\ATABValidity}{\emph{ATAB-Validity}}
\newcommand{\ATABAnnounce}{\emph{ATAB-Announcement}}
\newcommand{\ATABTermination}{\emph{ATAB-Termination}}
%\newcommand{\ATABFastAnnounce}{\emph{ATAB-Fast-Announcement}}

%% Command for observations.
\newtheorem{observation}{Observation}


%% HO ALGORITHM DEFINITIONS
\newconstruct{\FUNCTION}{\textbf{Function}}{\textbf{:}}{\ENDFUNCTION}{}

%% Uncomment the following four lines to remove remarks and visible traces of
%%    modifications in the document
%%\renewcommand{\sout}[1]{\relaxx}
%%\renewcommand{\uline}[1]{#1}
%% \renewcommand{\uwave}[1]{#1}
 \renewcommand{\note}[2][default]{\relax}


%% The following commands can be used to generate TR or Conference version of the paper
\newcommand{\tr}[1]{}
\renewcommand{\tr}[1]{#1}
\newcommand{\onlypaper}[1]{#1}
%\renewcommand{\onlypaper}[1]{}
%%%%%%%%%%%%%%%%%%%%%%%%%%%%%%%%%%%%%%%%%%%%%%%%%%%%%%%%%%%%%%%%%%%%%
%\pagestyle{plain}
%\pagestyle{empty}

%% IEEE tweaks
%\setlength{\IEEEilabelindent}{.5\parindent}
%\setlength{\IEEEiednormlabelsep}{.5\parindent}

\begin{document}
%
% paper title
% can use linebreaks \\ within to get better formatting as desired
\title{The latest gossip on BFT consensus\vspace{-0.7\baselineskip}}



\author{\IEEEauthorblockN{\large Ethan Buchman, Jae Kwon and Zarko Milosevic\\}
	\IEEEauthorblockN{\large Tendermint}\\
	%\\\vspace{-0.5\baselineskip}
	\IEEEauthorblockN{September 24, 2018}
}

% make the title area
\maketitle
\vspace*{0.5em}

\begin{abstract}
The paper presents Tendermint, a new protocol for ordering events in a distributed network under adversarial conditions. More commonly known as Byzantine Fault Tolerant (BFT) consensus or atomic broadcast, the problem has attracted significant attention in recent years due to the widespread success of blockchain-based digital currencies, such as Bitcoin and Ethereum, which successfully solved the problem in a public setting without a central authority. Tendermint modernizes classic academic work on the subject and simplifies the design of the BFT algorithm by relying on a peer-to-peer gossip protocol among nodes. 
\end{abstract}

%\noindent \textbf{Keywords:} Blockchain, Byzantine Fault Tolerance, State Machine %Replication

\section{Introduction} \label{sec:tendermint}

Consensus is one of the most fundamental problems in distributed computing. It
is important because of it's role in State Machine Replication (SMR), a generic
approach for replicating services that can be modeled as a deterministic state
machine~\cite{Lam78:cacm, Sch90:survey}. The key idea of this approach is that
service replicas start in the same initial state, and then execute requests
(also called transactions) in the same order; thereby guaranteeing that
replicas stay in sync with each other. The role of consensus in the SMR
approach is ensuring that all replicas receive transactions in the same order.
Traditionally, deployments of SMR based systems are in data-center settings
(local area network), have a small number of replicas (three to seven) and are
typically part of a single administration domain (e.g., Chubby
\cite{Bur:osdi06}); therefore they handle benign (crash) failures only, as more
general forms of failure (in particular, malicious or Byzantine faults) are
considered to occur with only negligible probability.  

The success of cryptocurrencies or blockchain systems in recent years (e.g.,
\cite{Nak2012:bitcoin, But2014:ethereum}) pose a whole new set of challenges on
the design and deployment of SMR based systems: reaching agreement over wide
area network, among large number of nodes (hundreds or thousands) that are not
part of the same administration domain, and where a subset of nodes can behave
maliciously (Byzantine faults). Furthermore, contrary to the previous
data-center deployments where nodes are fully connected to each other, in
blockchain systems, a node is only connected to a subset of other nodes, so
communication is achieved by gossip-based peer-to-peer protocols. 
The new requirements demand designs and algorithms that are not necessarily
present in the classical academic literature on Byzantine fault tolerant
consensus (or SMR) systems (e.g., \cite{DLS88:jacm, CL02:tcs}) as the primary 
focus was different setup. 

In this paper we describe a novel Byzantine-fault tolerant consensus algorithm
that is the core of the BFT SMR platform called Tendermint\footnote{The
	Tendermint platform is available open source at
	https://github.com/tendermint/tendermint.}. The Tendermint patform consists of
a high-performance BFT SMR implementation written in Go, a flexible interface
for
building arbitrary deterministic applications above the consensus, and a suite
of tools for deployment and management.  

The Tendermint consensus algorithm is inspired by the PBFT SMR
algorithm~\cite{CL99:osdi} and the DLS algorithm for authenticated faults (the
Algorithm 2 from \cite{DLS88:jacm}). Similar to DLS algorithm, Tendermint
proceeds in
rounds\footnote{Tendermint is not presented in the basic round model of
	\cite{DLS88:jacm}. Furthermore, we use the term round differently than in
	\cite{DLS88:jacm}; in Tendermint a round denotes a sequence of communication
	steps instead of a single communication step in \cite{DLS88:jacm}.}, where each
round has a dedicated proposer (also called coordinator or
leader) and a process proceeds to a new round as part of normal
processing (not only in case the proposer is faulty or suspected as being faulty
by enough processes as in PBFT).  
The communication pattern of each round is very similar to the "normal" case
of PBFT. Therefore, in preferable conditions (correct proposer, timely and
reliable communication between correct processes), Tendermint decides in three
communication steps (the same as PBFT). 

The major novelty and contribution of the Tendermint consensus algorithm is a
new termination mechanism. As explained in \cite{MHS09:opodis, RMS10:dsn}, the
existing BFT consensus (and SMR) algorithms for the partially synchronous
system model (for example PBFT~\cite{CL99:osdi}, \cite{DLS88:jacm},
\cite{MA06:tdsc}) typically relies on the communication pattern illustrated in
Figure~\ref{ch3:fig:coordinator-change} for termination. The
Figure~\ref{ch3:fig:coordinator-change} illustrates messages exchanged during
the proposer change when processes start a new round\footnote{There is no
	consistent terminology in the distributed computing terminology on naming
	sequence of communication steps that corresponds to a logical unit. It is
	sometimes called a round, phase or a view.}. It guarantees that eventually (ie.
after some Global Stabilization Time, GST), there exists a round with a correct
proposer that will bring the system into a univalent configuration.
Intuitively, in a round in which the proposed value is accepted
by all correct processes, and communication between correct processes is
timely and reliable, all correct processes decide.   


\begin{figure}[tbh!] \def\rdstretch{5} \def\ystretch{3} \centering
	\begin{rounddiag}{4}{2} \round{1}{~} \rdmessage{1}{1}{$v_1$}
		\rdmessage{2}{1}{$v_2$} \rdmessage{3}{1}{$v_3$} \rdmessage{4}{1}{$v_4$}
		\round{2}{~} \rdmessage{1}{1}{$x, [v_{1..4}]$}
		\rdmessage{1}{2}{$~~~~~~x, [v_{1..4}]$} \rdmessage{1}{3}{$~~~~~~~~x,
			[v_{1..4}]$} \rdmessage{1}{4}{$~~~~~~~x, [v_{1..4}]$} \end{rounddiag}
	\vspace{-5mm} \caption{\boldmath Proposer (coordinator) change: $p_1$ is the
		new proposer.} \label{ch3:fig:coordinator-change} \end{figure}  

To ensure that a proposed value is accepted by all correct
processes\footnote{The proposed value is not blindly accepted by correct
	processes in BFT algorithms. A correct process always verifies if the proposed
	value is safe to be accepted so that safety properties of consensus are not
	violated.}
a proposer will 1) build the global state by receiving messages from other
processes, 2) select the safe value to propose and 3) send the selected value
together with the signed messages
received in the first step to support it. The
value $v_i$ that a correct process sends to the next proposer normally
corresponds to a value the process considers as acceptable for a decision: 

\begin{itemize} \item in PBFT~\cite{CL99:osdi} and DLS~\cite{DLS88:jacm} it is
	not the value itself but a set of $2f+1$ signed messages with the same
	value id, \item in Fast Byzantine Paxos~\cite{MA06:tdsc} the value
	itself is being sent.  \end{itemize}

In both cases, using this mechanism in our system model (ie. high
number of nodes over gossip based network) would have high communication
complexity that increases with the number of processes: in the first case as
the message sent depends on the total number of processes, and in the second
case as the value (block of transactions) is sent by each process. The set of
messages received in the first step are normally piggybacked on the proposal
message (in the Figure~\ref{ch3:fig:coordinator-change} denoted with
$[v_{1..4}]$) to justify the choice of the selected value $x$. Note that
sending this message also does not scale with the number of processes in the
system.   

We designed a novel termination mechanism for Tendermint that better suits the
system model we consider. It does not require additional communication (neither
sending new messages nor piggybacking information on the existing messages) and
it is fully based on the communication pattern that is very similar to the
normal case in PBFT \cite{CL99:osdi}. Therefore, there is only a single mode of
execution in Tendermint, i.e., there is no separation between the normal and
the recovery mode, which is the case in other PBFT-like protocols (e.g.,
\cite{CL99:osdi}, \cite{Ver09:spinning} or \cite{Cle09:aardvark}). We believe
this makes Tendermint simpler to understand and implement correctly. 

Note that the orthogonal approach for reducing message complexity in order to
improve
scalability and decentralization (number of processes) of BFT consensus
algorithms is using advanced cryptography (for example Boneh-Lynn-Shacham (BLS)
signatures \cite{BLS2001:crypto}) as done for example in SBFT
\cite{Gue2018:sbft}.  

The remainder of the paper is as follows: Section~\ref{sec:definitions} defines
the system model and gives the problem definitions. Tendermint
consensus algorithm is presented in Section~\ref{sec:tendermint} and the
proofs are given in Section~\ref{sec:proof}. We conclude in
Section~\ref{sec:conclusion}.  





\section{Definitions}
\label{sec:definitions}


\subsection{Model}

We consider a system composed of $n$ server processes $\Pi = \{ 1, \dots, n\}$.
Server processes can be correct or faulty, where a faulty process can behave in an arbitrary way,
 i.e., we consider Byzantine faults. We assume that each process has some amount of voting power, and we denote with $N$ the total voting power of all processes in the system. 
Furthermore, we assume that the total voting power of faulty processes is bounded with a system parameter $f$.

Processes communicate by exchanging messages. We assume a model where processes are not part of the single administration domain; therefore we cannot enforce a direct network connectivity between all server processes.
Instead, we assume that each server process is connected to a small number of processes called peers, 
such that there is an indirect communication channel between all correct server processes. Communication between processes is established using gossip protocol.

We assume that processes communicate over wide-area network. Formally, we model the network communication using the partially synchronous system model~\cite{DLS88:jacm}, or rather a slightly weaker variant of this model: we assume that the system alternates between good periods (during which the system is synchronous) and bad periods (during which the system is asynchronous and messages can be lost). During good periods, there is a bound $\Delta$ such that all
communication among correct processes is reliable and $\Delta$-timely, i.e., if a correct process
$p$ sends message $m$ at time $t$ to a correct process $q$, then $q$ will receive $m$
before $t+\Delta$. Note that as we do not assume direct communication channels among all correct processes, this implies that before the message $m$ reaches $q$, it will pass through a number of 
correct servers that will forward the message $m$ using gossip protocol towards $q$. 
Messages among correct processes can be
delayed, dropped or duplicated during bad periods. Spoofing/impersonation attacks are assumed to be impossible also during bad periods.
The bound $\Delta$ is a system parameter whose value is not required to be known for the safety of the algorithm presented. However, the timing bounds derived for our algorithm, and thus the guaranteed latency requires knowledge of $\Delta$.

We assume that process steps (which might include sending and receiving messages) take zero time.
Processes are equipped with clocks so they can measure local timeouts.

We assume integrity of direct communication channels between peers, that is if a process $p$ received a message $m$ from process $q$, then $q$ sent message $m$ to $p$ before. All protocol messages are signed, i.e., when a correct process $q$ receives a signed message $m$ from its peer, the process $q$ can verify who was the original message sender.

The details of the Tendermint gossip protocol will be discussed in a separate technical report. For the sake of this report it is sufficient to assume the following properties provided by the gossip protocol (in addition to the properties about partial synchrony of the network stated above):

\begin{itemize}
	\item Messages that are being gossiped come from the consensus layer. We can think about consensus protocol as a content creator, which defines what messages should be disseminated using the gossip protocol. A correct process creates the message for dissemination either i) explicitly, by invoking \emph{send} function as part of the consensus protocol or ii) implicitly, by receiving a message from some other process. Note that in the case ii) gossiping of messages is implicit, i.e., it happens without explicit send clause in the consensus algorithm whenever a correct process receives some messages in the consensus algorithm\footnote{If a message is received by a correct process at the consensus level then it is considered valid from the protocol point of view, i.e., it has a correct signature, a proper message structure and a valid height and round number.}. 
    \item Processes keep resending messages (in case of failures or message loss) until all its peers get them. This ensures that every message sent or received by a correct process is eventually received by all correct processes. 
\end{itemize}


 \subsection{Consensus}
 \label{sec:consensus}

 \newcommand{\propose}{\mathsf{propose}}
 \newcommand{\decide}{\mathsf{decide}}

We consider a variant of the Byzantine consensus problem called Validity Predicate-based Byzantine consensus that is motivated by blockchain systems~\cite{GLR17:red-belly-bc}. The problem is defined by an agreement, a termination, and a validity
property.

 \begin{itemize}
 \item \emph{Agreement:} No two correct processes decide on different values.
 \item \emph{Termination:} All correct processes eventually decide on a value.
 \item \emph{Validity:} A decided value is valid, i.e., it satisfies the predefined predicate denoted \emph{valid()}.
 \end{itemize}

This variant of the Byzantine consensus problem has an application-specific valid() predicate to indicate whether a value is valid. In the context of blockchain systems, for example, a value is not valid if it does not contain an appropriate hash of the last value (block) added to the blockchain.



\section{Tendermint consensus algorithm}
\label{sec:tendermint}

\newcommand\Disseminate{\textbf{Disseminate}}

\newcommand\Proposal{\mathsf{PROPOSAL}}
\newcommand\ProposalPart{\mathsf{PROPOSAL\mbox{-}PART}}
\newcommand\PrePrepare{\mathsf{INIT}}
\newcommand\Prevote{\mathsf{PREVOTE}}
\newcommand\Precommit{\mathsf{PRECOMMIT}}
\newcommand\Decision{\mathsf{DECISION}}

\newcommand\ViewChange{\mathsf{VC}}
\newcommand\ViewChangeAck{\mathsf{VC\mbox{-}ACK}}
\newcommand\NewPrePrepare{\mathsf{VC\mbox{-}INIT}}
\newcommand\coord{\mathsf{proposer}}

\newcommand\newHeight{newHeight}
\newcommand\newRound{newRound}
\newcommand\nil{nil}
\newcommand\id{id}
\newcommand{\propose}{propose}
\newcommand\prevote{prevote}
\newcommand\prevoteWait{prevoteWait}
\newcommand\precommit{precommit}
\newcommand\precommitWait{precommitWait}
\newcommand\commit{commit}

\newcommand\timeoutPropose{timeoutPropose}
\newcommand\timeoutPrevote{timeoutPrevote}
\newcommand\timeoutPrecommit{timeoutPrecommit}
\newcommand\proofOfLocking{proof\mbox{-}of\mbox{-}locking}

\begin{algorithm}[htb!]
\def\baselinestretch{1}
\scriptsize\raggedright
\begin{algorithmic}[1]
	\SHORTSPACE
\INIT{}
\STATE $h_p := 0$ \COMMENT{current height, or consensus instance we are currently executing}
\STATE $round_p := 0$   \COMMENT{current round number} 
\STATE $step_p  \in \set{\propose=0, \prevote=1, \prevoteWait=2, \precommit=3}$, initially $\propose$  
\STATE $decision_p[] := nil$ for every element of the array
\STATE $lockedValue_p := nil$
\STATE $lockedRound_p := -1$ 
\STATE $validValue_p := nil$
\STATE $validRound_p := -1$
%\STATE ~~$\timeoutPropose := initTimeoutPropose + r*timeoutDelta$
%\STATE ~~$\timeoutPrevote := initTimeoutPrevote + r*timeoutDelta$
%\STATE $\timeoutPrecommit := initTimeoutPrecommit + r*timeoutDelta$
\ENDINIT
\SHORTSPACE
\STATE \textbf{upon} start \textbf{do}  $StartRound(0)$
\SHORTSPACE
\FUNCTION{$StartRound(round)$} \label{line:tab:startRound}
	\STATE	$round_p \assign round$
	\STATE	$step_p \assign \propose$
	\IF{$\coord(h_p, round_p) = p$} 
		\IF{$validValue_p \neq \nil$} \label{line:tab:isThereLockedValue}
			\STATE $proposal \assign validValue_p$
		\ELSE
			\STATE $proposal \assign getValue()$ \label{line:tab:getValidValue}
		\ENDIF 	  
		\STATE \Broadcast\ $\li{\Proposal,h_p, round_p, proposal, validRound_p}$  \label{line:tab:send-proposal}
	\ELSE
		\STATE \textbf{after} $\timeoutPropose$ execute $OnTimeoutPropose(h_p, round_p)$ 
	\ENDIF
\ENDFUNCTION

\SPACE
\UPON{$\li{\Proposal,h_p,round_p, v, r}$ \From\ $\coord(h_p,round_p)$ \With\ $state_p = \propose$} \label{line:tab:recvProposal}			
 \IF{$!valid(v) \vee (lockedRound_p > r  \wedge lockedValue_p \neq v$)}  \label{line:tab:acceptProposal1}		
 	\STATE \Broadcast \ $\li{\Prevote,h_p,round_p,\nil}$  \label{line:tab:prevote-nil}	
 	\STATE $step_p \assign \prevote$ \label{line:tab:setStateToPrevote1} 
 \ELSIF{$valid(v) \wedge (lockedRound_p = -1  \vee lockedValue_p = v$)} \label{line:tab:accept-proposal-2}
 	\STATE \Broadcast \ $\li{\Prevote,h_p,round_p,id(v)}$  \label{line:tab:prevote-proposal}	
 	\STATE $step_p \assign \prevote$ \label{line:tab:setStateToPrevote2} 
 \ENDIF
\ENDUPON

\SPACE
\UPON{$\li{\Proposal,h_p,round_p, v, r}$ \From\ $\coord(h_p,round_p)$ \textbf{AND} $2f+1$ $\li{\Prevote,h_p, r,id(v)}$  \With\
	$step_p = \propose$} \label{line:tab:acceptProposal}
	\IF{$r \ge lockedRound_p \wedge r < round_p \wedge valid(v)$} \label{line:tab:cond-prevote-higher-proposal}	
		\STATE \Broadcast \ $\li{\Prevote,h_p,round_p,id(v)}$  \label{line:tab:prevote-higher-proposal}	
		\STATE $step_p \assign \prevote$ \label{line:tab:setStateToPrevote3} 		 
	\ENDIF
\ENDUPON

\SPACE
\UPON{$2f+1$ $\li{\Prevote,h_p, round_p,*}$ \With\ $state_p = \prevote$} \label{line:tab:recvAny2/3Prevote}
	\STATE \textbf{after} $\timeoutPrevote$ execute $OnTimeoutPrevote(h_p, round_p)$ \label{line:tab:timeoutPrevote}
		\STATE $step_p \assign \prevoteWait$ \label{line:tab:setStateToPrevoteWait} 
\ENDUPON

\SPACE
\UPON{$\li{\Proposal,h_p,round_p, v, *}$ \From\ $\coord(h_p,round_p)$ \textbf{AND} $2f+1$ $\li{\Prevote,h_p, round_p,id(v)}$  \With\ $valid(v)$} \label{line:tab:recvPrevote}
	\IF{$step_p = \prevote$}	
	\STATE $lockedValue_p \assign v$                \label{line:tab:setLockedValue}
	\STATE $lockedRound_p \assign round_p$   \label{line:tab:setLockedRound} 
	\STATE \Broadcast \ $\li{\Precommit,h_p,round_p,id(v))}$  \label{line:tab:precommit-v}	
	\STATE $step_p \assign \precommit$ \label{line:tab:setStateToCommit}
	\ENDIF
	\STATE $validValue_p \assign v$                          \label{line:tab:setValidRound}
	\STATE $validRound_p \assign round_p$             \label{line:tab:setValidValue}
\ENDUPON

\SHORTSPACE
\UPON{$2f+1$ $\li{\Prevote,h_p,round_p, \nil}$ \With\ $state_p = \prevote$}
	\STATE \Broadcast \ $\li{\Precommit,h_p,round_p, \nil}$   \label{line:tab:precommit-v-1}
	\STATE $step_p \assign \precommit$
\ENDUPON

\SPACE
\UPON{$2f+1$ $\li{\Precommit,h_p,round_p,*}$ for the first time} \label{line:tab:startTimeoutPrecommit}
	\STATE \textbf{after} $\timeoutPrecommit$ execute $OnTimeoutPrecommit(h_p, round_p)$
\ENDUPON 

\SPACE
\UPON{$\li{\Proposal,h_p,r, v, *}$ \From\ $\coord(h_p,r)$ \textbf{AND} $2f+1$ $\li{\Precommit,h_p,r,id(v)}$ \With\ $decision_p[h_p] = \nil$} \label{line:tab:onDecideRule} 
\IF{$valid(v)$} \label{line:tab:validDecisionValue}
	\STATE $decision_p[h_p] = v$   \label{line:tab:decide} 
	\STATE$h_p \assign h_p + 1$  \label{line:tab:increaseHeight} 
	\STATE reset $lockedRound_p$, $lockedValue_p$ to init values and empty message log 
	\STATE $StartRound(0)$   	
\ENDIF
\ENDUPON

\SHORTSPACE
\UPON{$f+1$ $\li{*,h_p,round, *, *}$ \textbf{with} $round > round_p$} \label{line:tab:skipRounds}
\STATE $StartRound(round)$ \label{line:tab:nextRound2}
\ENDUPON

\SHORTSPACE
\FUNCTION{$OnTimeoutPropose(height,round)$} \label{line:tab:onTimeoutPropose}
\IF{$height = h_p \wedge round = round_p \wedge step_p = \propose$}
\STATE \Broadcast \ $\li{\Prevote,h_p,round_p, \nil}$  \label{line:tab:prevote-nil-on-timeout}	
\STATE $step_p \assign \prevote$
\ENDIF	
\ENDFUNCTION

\SHORTSPACE
\FUNCTION{$OnTimeoutPrevote(height,round)$} \label{line:tab:onTimeoutPrevote}
\IF{$height = h_p \wedge round = round_p \wedge step_p = \prevoteWait$}
\STATE \Broadcast \ $\li{\Precommit,h_p,round_p,\nil}$   \label{line:tab:precommit-nil-onTimeout}
\STATE $step_p \assign \precommit$
\ENDIF	
\ENDFUNCTION

\SHORTSPACE
\FUNCTION{$OnTimeoutPrecommit(height,round)$} \label{line:tab:onTimeoutPrecommit}
\IF{$height = h_p \wedge round = round_p$}
\STATE $StartRound(round_p + 1)$ \label{line:tab:nextRound} 
\ENDIF	
\ENDFUNCTION	
\end{algorithmic}
\caption{Tendermint consensus algorithm}
\label{alg:tendermint}
\end{algorithm}

In this section we present Tendermint Byzantine fault-tolerant consensus algorithm. 
The code of the algorithm is given as Algorithm~\ref{alg:tendermint}. We present the algorithm as a set of \emph{upon rules} that are executed atomically\footnote{In case several rules are active at the same time, the first rule to be executed is picked randomly. The correctness of the algorithm does not depend on the order in which rules are executed.}. We assume that processes exchange protocol messages using gossip protocol and received and sent messages at every process are stored in the local \emph{message log}. An upon rule is triggered once the message log contains messages such that the corresponding condition evaluates to $\tt{true}$. The condition that assumes a reception of $X$ messages of particular type and content, denotes set of messages whose senders have aggregate voting power at least equal to $X$. The variables with index $p$ are process local state variables, while variables without index $p$ are value placeholders. The sign $*$ denotes any value.    

We denote with $n$ the total voting power of processes in the system, and we assume that the total voting power of faulty processes in the system is bounded with a system parameter $f$. 
The algorithm assumes that $n > 3f$, i.e., it requires that the total voting power of faulty processes is smaller than one third of the total voting power. For simplicity we present the algorithm for the case $n = 3f + 1$.

The algorithm proceeds in rounds, where each round has a dedicated \emph{proposer}. The assignment scheme of rounds to proposers is known to all processes and is given as a function $\coord_p(h_p, round_p)$, returning the proposer for the round $round_p$ in the consensus instance $h_p$ at the process $p$. We assume that the proposer selection function is weighted round-robin, where processes are rotated proportional to its voting power\footnote{A validator with more voting power is selected more frequently, proportional to its power. More precisely, during a sequence of rounds of size $n$, every process is proposer in a number of rounds equal to its voting power.}. 

Processes exchange the following messages in Tendermint: $\Proposal$, $\Prevote$ and $\Precommit$. The 
$\Proposal$ message is used by the proposer of the current round to suggest a potential decision value, while 
$\Prevote$ and $\Precommit$ are votes for a proposed value. Tendermint belongs to the class 3 of consensus algorithms (like PBFT \cite{CL02:tcs} and DLS \cite{DLS88:jacm}) according to the classification of consensus algorithms from \cite{RMS10:dsn}, so it requires two voting steps (three communication exchanges in total) to decide a value. The Tendermint consensus algorithm is designed for the blockchain context where a value to decide is a block of transactions (so potentially quite big value). Therefore, in the Algorithm \ref{alg:tendermint} (similar as in \cite{CL02:tcs}) we are explicit about sending a value (block of transactions) and a small, constant size value id (a unique value identifier, normally a hash of the value, i.e., if $\id(v) = \id(v')$, then $v=v'$). The $\Proposal$ message is the only one carrying the value; $\Prevote$ and $\Precommit$ messages carry the value id. 
A correct process decides on a value $v$ in Tendermint upon receiving the $\Proposal$ for $v$ and $2f+1$ voting-power equivalent $\Precommit$ messages for $\id(v)$ in some round $r$. In order to send $\Precommit$ message for $v$ in a round $r$, a correct process waits to receive the $\Proposal$ and $2f+1$ the corresponding $\Prevote$ messages in the round $r$. Otherwise it sends $\Precommit$ message with a special $\nil$ value.  This ensures that correct processes can $\Precommit$ only a single value (or $\nil$) in a round. 
As a proposer might be a faulty process, the proposed value is treated by correct processes as a suggestion (it is not blindly accepted), and a correct process tells others if it accepted the $\Proposal$ for value $v$ by sending $\Prevote$ message for $\id(v)$; otherwise it sends $\Prevote$ message with a special $\nil$ value. 

Every process maintains the following variables in the Algorithm \ref{alg:tendermint}: $step$, $lockedValue$, $lockedRound$, $validValue$ and $validRound$. The $step$ denotes the current state of the internal Tendermint state machine, i.e., it reflects the stage of the algorithm execution in the current round. The $lockedValue$ stores the most recent value (with respect to a round number) for which a $\Precommit$ message has been sent. The $lockedRound$ is the last round in which process sent $\Precommit$ message that is not $\nil$. We also say that a correct process locks a value $v$ in a round $r$ by setting $lockedValue = v$ and $lockedRound = r$ before sending $\Precommit$ message for $\id(v)$. As a correct process can decide a value $v$ only if $2f+1$ $\Precommit$ messages for $\id(v)$ are received, this implies that a possible decision value is a value that is locked by at least $f+1$ voting power equivalent of correct processes. Therefore, any value $v$ for which $\Proposal$ and $2f+1$ the corresponding $\Prevote$ messages are received in some round $r$ is a \emph{possible decision} value. The role of $validValue$ variable is to store the most recent possible decision value; the $validRound$ is the last round in which $validValue$ is updated. Apart from those variables, a process also stores a current height ($h_p$) and a current round number ($round_p$) that are attached to every message. Finally a process also stores an array of decisions (Tendermint assumes a sequence of consensus instances) $decision_p$. 

Every round starts by a proposer suggesting a value with $\Proposal$ message (see line \ref{line:tab:send-proposal}). In the initial round of each consensus instance (called \emph{height} in Tendermint), a proposer is free to chose the value to suggest. In the Algorithm~\ref{alg:tendermint}, a correct process obtains a value to propose using an external function    
$getValue()$ that returns a valid value to propose. In the following rounds, a correct proposer will suggest a new value only if $validValue \neq \nil$; otherwise $validValue$ is proposed (see lines~\ref{line:tab:isThereLockedValue}-\ref{line:tab:getValidValue}). Note that if a correct proposer $p$ sends $validValue$ with the $validRound$ in the $\Proposal$, this implies that the process $p$ received $\Proposal$ and the corresponding $2f+1$ $\Prevote$ messages for $validValue$ in the round $validRound$. In addition to the value proposed, the $\Proposal$ message also contains the $validRound$ so other processes are informed about the last round in which the proposer observed $validValue$ as a possible decision value.  
If a correct process sends $\Proposal$ message with $validValue$ ($validRound > -1$) at time $t > GST$, by the \emph{Gossip communication} property, the corresponding $\Proposal$ and the $\Prevote$ messages will be received by all correct processes before time $t+\Delta$. Therefore, all correct processes will be able to verify the correctness of the suggested value as it supported by the $\Proposal$ and the corresponding $2f+1$ voting power equivalent $\Prevote$ messages.   

A correct process $p$ accepts the proposal for a value $v$  (send $\Prevote$ for $id(v)$) if an external \emph{valid} function returns $true$ for the value $v$,
and if $p$ hasn't locked any value ($lockedRound = -1$) or $p$ has locked the value $v$ ($lockedValue = v$); see the line \ref{line:tab:accept-proposal-2}. 
In case the proposed pair is $(v,r)$ and a correct process $p$ has locked some other value ($v' \neq v$), it will accept $v$ only if $v'$ was more recent possible decision value\footnote{As explained above, the possible decision value in a round $r$ is the one for which $\Proposal$ and the corresponding $2f+1$ $\Prevote$ messages are received for the round $r$.} in the round higher than $lockedRound_p$ ($r > lockedValue_p$).   
Otherwise, a correct process will reject the proposal by sending $\Prevote$ message with $\nil$ value. A correct process will send $\Prevote$ message with $\nil$ value also in case $\timeoutPropose$ expired (it is started when a correct process starts a new round) and a process has not sent $\Prevote$ message in the current round yet (see the line \ref{line:tab:onTimeoutPrevote}). 

If a correct process receives $\Proposal$ message for some value $v$ and $2f+1$ $\Prevote$
messages for $\id(v)$, then it sends $\Precommit$ message with $\id(v)$. Otherwise, it sends $\Precommit$ $\nil$. A correct process will send $\Precommit$ message with $\nil$ value also in case $\timeoutPrevote$ expired (it is started when a correct process sent $\Prevote$ message and received any $2f+1$ $\Prevote$ messages)  and a process has not sent $\Precommit$ message in the current round yet (see the line \ref{line:tab:onTimeoutPropose}). 
A correct process decides on some value $v$ if it receives in some round $r$ $\Proposal$ message for $v$ and $2f+1$ $\Precommit$ messages with $\id(v)$ (see the line \ref{line:tab:decide}).  
To prevent algorithm for blocking and waiting forever for this condition to be true, the Algorithm \ref{alg:tendermint} relies on $\timeoutPrecommit$. It is triggered after a process receives any set of $2f+1$ $\Precommit$ messages for the current round. If the $\timeoutPrecommit$ expires and a process has not decided yet, the process starts the next round (see the line \ref{line:tab:onTimeoutPrecommit}).  
The timeouts are increased with every new round $r$, i.e, $timeoutX(r) = initTimeoutX + r*timeoutDelta$, and are reset for every new height (consensus instance). Increasing the timeout values ensures that eventually communication between correct processes in a round is reliable and timely, so correct processes can decide. 

\subsection{Termination mechanism}

Tendermint ensures termination by the novel mechanism that benefits from the gossip based nature of communication (see \emph{Gossip communication} property).  
 It requires managing two additional variables, $validValue$ and $validRound$ that are then used by the proposer during the propose step as explained above.   
The $validValue$ and $validRound$ are updated to $v$ and $r$ by a correct process in a round $r$ when the process receives valid $\Proposal$ message for the value $v$ and the corresponding $2f+1$ $\Prevote$ messages for $id(v)$ in the round $r$ (see the rule at line~\ref{line:tab:recvPrevote}).

We now give briefly the intuition how managing and proposing $validValue$ and $validRound$ ensures termination and leave the formal treatment for Section~\ref{sec:proof}.  First thing to note is that at the beginning of every round during the system execution there is at least single correct process $c$ such that $validValue_c$ and $validRound_c$ are acceptable by every correct process. This is true as at there exists a correct process $c$ such that for every other correct process $p$, $validRound_c > lockedRound_p$ or $validValue_c = lockedValue_p$. This is true as $c$ is the process that has locked value in the most recent round among all correct processes (or no correct process locked any value). Therefore, if we assume that at the beginning of round $r_0$ in the good period (after GST and timeouts are big enough so communication between correct processes is timely and reliable) $c$ is that process and we assume that $c$ will be proposer in a round $r_1 \ge r_0$ (as proposer function is weighted round robin), and we assume that no correct process locks a value in rounds between, then the $validValue_c$ and $validRound_c$ will be acceptable by all correct processes. Note that we rely here on the \emph{Gossip communication} property that will ensure that $\Prevote$ messages that are received by $c$ at line~\ref{line:tab:recvPrevote} when $c$ updated $validValue_c$ in the round $validRound_c$ are received by all correct processes before $\timeoutPropose$ expires in the round $r_1$ (see the rule at line \ref{line:tab:acceptProposal}).   

The second thing to note is that during good period, because of the \emph{Gossip communication} property, if a correct process $p$ locks a value $v$ is some round $r$, all correct processes will update $validValue$ to $v$ and $validRound$ to $r$ before the end of the round $r$ (we prove this formally in the Section~\ref{sec:proof}). The intuition is that messages that led to $p$ locking a value $v$ in the round $r$ will be gossiped to all correct processes before the end of the round $r$, so it will update $validValue$ and $validRound$ (the lineline~\ref{line:tab:recvPrevote}). Therefore, if a correct process locks some value during good period, $validValue$ and $validRound$ are updated by all correct processes so that the value proposed in the following rounds will be acceptable by all correct processes. And last thing to note is that it could happen that during good period, no correct process locks a value, but some correct process $q$ updates $validValue$ and $validRound$ during some round. As no correct process locks a value in this case, $validValue_q$ and $validRound_q$ will also be acceptable by all correct processes as $validRound_q > lockedRound_c$ for every correct process $c$ and as the \emph{Gossip communication} property ensures that the corresponding $\Prevote$ messages that $q$ received in the round $validRound_q$ are received by all correct process $\Delta$ time later. 

Therefore, updating $validValue$ and $validRound$ variables, and the \emph{Gossip communication} property, together ensures that eventually, during the good period, exists a round with a correct proposer whose proposed value will be accepted by all correct processes, and all correct processes will terminate in that round. Note that this mechanism, contrary to the common termination mechanism illustrated in the Figure~\ref{ch3:fig:coordinator-change}, does not require exchanging any additional information in addition to messages already sent as part of what is normally being called "normal" case.     

\section{Formal proof of Tendermint consensus algorithm}
\label{sec:proof}
\section{Conclusion}
\label{sec:conclusion}

We have proposed a new Byzantine-fault tolerant consensus algorithm that is the core of the Tendermint BFT SMR platform. The algorithm
is designed for the wide area network with high number of mutually distrusted nodes that communicate over gossip based peer-to-peer network. It has only a single mode of execution and the communication pattern is very similar to the "normal" case of the state-of-the art PBFT algorithm. The algorithm ensures termination with a novel mechanism that takes advantage of the gossip based communication between nodes.
The proposed algorithm and the proofs are simple and elegant, and we believe that this makes it easier to understand and implement correctly.   

\section*{Acknowledgment}

We would like to thank...

\bibliographystyle{IEEEtran}
\bibliography{lit}

%\appendix

\end{document}
